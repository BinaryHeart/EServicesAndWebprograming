\documentclass[12pt]{article}
\usepackage{fullpage}
\usepackage[swedish]{babel}
\usepackage[utf8]{inputenc} % åäö
%\usepackage[T1]{fontenc}
\usepackage{graphicx}
\usepackage{hyperref}
\usepackage{xcolor}
\usepackage{listings}
\usepackage{enumitem}

% No numbering
\setcounter{secnumdepth}{0}

% Counters for tasks & questions
\newcounter{taskcounter}
\setcounter{taskcounter}{0}
\newcounter{stepcounter}
\setcounter{stepcounter}{0}
\newcounter{questioncounter}
\setcounter{questioncounter}{0}

% Exercises
\newcommand{\exercise}[1]{
  \refstepcounter{taskcounter}
  \addcontentsline{toc}{subsection}{Uppgift \thetaskcounter{} #1}
  \vspace{1em}~
  \\\normalfont{\large{\bfseries{\hspace{0.5em}Uppgift \thetaskcounter \hspace{1em}#1}}}\\\\
}

% Remove date
\date{}

\hypersetup{
  colorlinks = true,
  linkcolor = blue,
  citecolor = red
}

\lstset{
  language=[Sharp]C,
  basicstyle=\color[rgb]{0.3,0.3,0.3}\ttfamily,
  keywordstyle=\color[rgb]{0,0.5,0.5},
  numberstyle=\color[rgb]{0.7,0.7,0.7},
  commentstyle=\color[rgb]{0.1,0.5,0.1},
  stringstyle=\color[rgb]{0.6,0.1,0.5},
  backgroundcolor=\color[rgb]{0.95,0.95,0.95},
  showstringspaces=false,
  numbers=left,
  breaklines,
  breakatwhitespace,
}

\title{ Labb 3 }

\author{ E-tjänster och webbprogrammering 7.5 hp VT-14 }
\begin{document}
\maketitle
\vspace{-2em}
%\tableofcontents



\section{Introduktion}
I denna laboration kommer vi att utöka gästboken med att man skall kunna ge feedback på gästboken. Det betyder att varje inlägg bör ha en länk eller liknande för att besvara ett inlägg. Detta kommer innebära att det kommer kräva en algoritm för att kunna generera och sortera att inläggen kommer i rätt ordning. Designen för att hur er dagbok ser ut kommer ni själv få välje men kräver ändå lite av gräsnssnittet för att det skall se bra ut. http://designm.ag/designer-showcase/well-designed-threaded-comments-sections-in-webpages/



\section{Föreslagen struktur}
Denna laboration består av 4 faser men kommer endast resultera till ett program. Denna laboration kräver ingen inlämning men för att få en inblick om vad som krävs ges ett förslag på hur ni kan strukturera era dokument.
  \begin{itemize}
    \item labb1\_fornamn\_efternamn.zip
      \begin{itemize}
        \item Laboration1 (mapp)
          \begin{itemize}
            \item index.php 
            \item images (mapp med bilder)
            \item js (mapp med javscriptfiler)
	   \item css (mapp med stylesheets)
          \end{itemize}
    \end{itemize}
  \end{itemize}




\pagebreak
\section{ }
Nedan följer uppgifterna som resulterar i inlämningen ovan.

  \exercise{Designa databasen}
 Ni bör nu i varje inlägg kunna lägga till vem inlägget tillhör, enklast är att ha en förälder till noden. 


 \exercise{Hämta och spara data}
  Skapa nu en funktion för att kunna besvara inlägget, antingen skapar ni en ny sida eller att man skickas till det formulär ni redan har och sedan lägger in det i databasen. 

 \exercise{Smyckning}
Nu är det dags att smycka er sida så att man kan tydligt se vilken som tillhör vem. 




\end{document}