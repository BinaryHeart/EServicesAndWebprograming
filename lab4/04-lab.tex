\documentclass[12pt]{article}
\usepackage{fullpage}
\usepackage[swedish]{babel}
\usepackage[utf8]{inputenc} % åäö
%\usepackage[T1]{fontenc}
\usepackage{graphicx}
\usepackage{hyperref}
\usepackage{xcolor}
\usepackage{listings}
\usepackage{enumitem}

% No numbering
\setcounter{secnumdepth}{0}

% Counters for tasks & questions
\newcounter{taskcounter}
\setcounter{taskcounter}{0}
\newcounter{stepcounter}
\setcounter{stepcounter}{0}
\newcounter{questioncounter}
\setcounter{questioncounter}{0}

% Exercises
\newcommand{\exercise}[1]{
  \refstepcounter{taskcounter}
  \addcontentsline{toc}{subsection}{Uppgift \thetaskcounter{} #1}
  \vspace{1em}~
  \\\normalfont{\large{\bfseries{\hspace{0.5em}Uppgift \thetaskcounter \hspace{1em}#1}}}\\\\
}

% Remove date
\date{}

\hypersetup{
  colorlinks = true,
  linkcolor = blue,
  citecolor = red
}

\lstset{
  language=[Sharp]C,
  basicstyle=\color[rgb]{0.3,0.3,0.3}\ttfamily,
  keywordstyle=\color[rgb]{0,0.5,0.5},
  numberstyle=\color[rgb]{0.7,0.7,0.7},
  commentstyle=\color[rgb]{0.1,0.5,0.1},
  stringstyle=\color[rgb]{0.6,0.1,0.5},
  backgroundcolor=\color[rgb]{0.95,0.95,0.95},
  showstringspaces=false,
  numbers=left,
  breaklines,
  breakatwhitespace,
}

\title{ Labb 4 }

\author{ E-tjänster och webbprogrammering 7.5 hp VT-14 }
\begin{document}
\maketitle
\vspace{-2em}
%\tableofcontents



\section{Introduktion}
I denna laboration kommer vi att arbeta i två pass. I det första kommer vi att refaktorera vår applikation (alltså resultatet av tidigare labbar). För att öka nivån av ``kontroll'', struktur och läsbarhet. Vi kommer att refaktorera vår \texttt{HTML}, \texttt{CSS}, \texttt{PHP}, \texttt{JavaScript} såväl som vår \texttt{SQL}. Med andra ord all kod vi skrivit :)

I det andra passet av denna laboration kommer vi att arbeta med att bygga om vår sida till en \href{http://en.wikipedia.org/wiki/Single-page_application}{Single Page Application}. Vi kommer alltså använda det tillvägagångssätt som ofta benämns \texttt{AJAX} för att se till att användaren kan interagera med sidan som vanligt, men utan att någon sida någonsin fullständigt ``laddas om''.

\section{Föreslagen struktur}
Denna laboration består av 2 faser. Dessa är inte separata uppgifter, utan föreslagna steg för att bygga ovan nämnd applikation. Som alltså är en vidarutveckling av den applikation vi byggt i labb 1, 2 och 3. Denna laboration kräver ingen inlämning. Men för att få en uppfattning om hur ett färdigt projekt efter denna labb skulle kunna se ut -- beakta nedan filstruktur.
\begin{itemize}
  \item \texttt{labb4-fornamn-efternamn} (mapp)
  \begin{itemize}
    \item \texttt{index.php}
    \item \texttt{include} (mapp)
      \begin{itemize}
        \item \texttt{db-connect.php}
        \item \texttt{db-queries.php}
        \item \texttt{process-register.php}
        \item \texttt{process-login.php}
        \item \texttt{process-post-comment.php}
      \end{itemize}
    \item \texttt{assets} (mapp)
    \begin{itemize}
      \item \texttt{img} (mapp med bilder)
      \item \texttt{js } (mapp med javscriptfiler)
      \item \texttt{css} (mapp med stylesheets)
    \end{itemize}
  \end{itemize}
\end{itemize}

Hur många filer ni har i varje mapp beror förstås på vilken strategi ni väljer att lösa problemet med. Samt vilka refaktoreringar ni faktiskt genomför. Läs vidare i instruktionerna för klarare förståelse.



\pagebreak
\section{Uppgifter}
Nedan följer uppgifterna som resulterar i inlämningen ovan.

\exercise{Refaktorering}
Refaktorera din kod så att vi \textbf{åtminstone} uppfyller nedan.

\begin{itemize}
  \item Organisera dina filer i en tydlig och konsistent mappstruktur.
  \item Eliminera duplikation av liknande kod \textbf{inom ett dokument} (i.e.generalisera).
  \item Eliminera duplikation av liknande kod \textbf{emellan dokument} (i.e. generalisera).
  \item Bryt ut långa block av procedurell kod till \textbf{metoder}.
  \item Låt alla \texttt{variabler}, \texttt{metoder} och \texttt{klasser} ha beskrivande och tydliga namn.
  \item Bryt ut varje databas-query till en egen \texttt{metod}. Använd metodargument för att skicka värden.
  \item Om flera databas-query-metoder är liknande. Försök slå ihop dem.
  \item Separera ``printing'' och ``processing''. Låt \textbf{endast} de ``yttersta'' filerna hantera utskrivning av \texttt{html}. Låt inte ``process-filerna'' hantera \texttt{html}. Låt de istället hantera data-processning och business logic. Se förslagen inlämningsstruktur (högst upp) för ett förslag på denna indelning.
  \item Blanda \texttt{html} och \texttt{php} på ett tydligt och konsistent sätt.
\end{itemize}




\exercise{Single-page application}
När vi ifrån en webbsida, navigerar till en annan sida, så behöver webbläsaren ``ladda om''. Detta gäller alltså förstås då även när vi t.ex. postar formulär. I vårat fall -- när användaren postar kommentarer.

Den teknikfamilj som ofta benämns \texttt{AJAX} används alltså för att kunna skicka data fram och tillbaka emellan användarens klient och servern utan att användarens webbläsare ska behöva ladda om sidan.

Din uppgift är nu att se till att eliminera alla sidomladdningar i din applikation. Om användaren med andra ord postar en kommentar, så ska denna kommentar postas utan att hela sidan laddas om. Ett anrop till servern, och ett hanterande av ett response måste alltså göras i bakgrunden med hjälp av \texttt{JavaScript}.




\end{document}