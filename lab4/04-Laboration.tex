\documentclass[12pt]{article}
\usepackage{fullpage}
\usepackage[swedish]{babel}
\usepackage[utf8]{inputenc} % åäö
%\usepackage[T1]{fontenc}
\usepackage{graphicx}
\usepackage{hyperref}
\usepackage{xcolor}
\usepackage{listings}
\usepackage{enumitem}

% No numbering
\setcounter{secnumdepth}{0}

% Counters for tasks & questions
\newcounter{taskcounter}
\setcounter{taskcounter}{0}
\newcounter{stepcounter}
\setcounter{stepcounter}{0}
\newcounter{questioncounter}
\setcounter{questioncounter}{0}

% Exercises
\newcommand{\exercise}[1]{
  \refstepcounter{taskcounter}
  \addcontentsline{toc}{subsection}{Uppgift \thetaskcounter{} #1}
  \vspace{1em}~
  \\\normalfont{\large{\bfseries{\hspace{0.5em}Uppgift \thetaskcounter \hspace{1em}#1}}}\\\\
}

% Remove date
\date{}

\hypersetup{
  colorlinks = true,
  linkcolor = blue,
  citecolor = red
}

\lstset{
  language=[Sharp]C,
  basicstyle=\color[rgb]{0.3,0.3,0.3}\ttfamily,
  keywordstyle=\color[rgb]{0,0.5,0.5},
  numberstyle=\color[rgb]{0.7,0.7,0.7},
  commentstyle=\color[rgb]{0.1,0.5,0.1},
  stringstyle=\color[rgb]{0.6,0.1,0.5},
  backgroundcolor=\color[rgb]{0.95,0.95,0.95},
  showstringspaces=false,
  numbers=left,
  breaklines,
  breakatwhitespace,
}

\title{ Labb 4 }

\author{ E-tjänster och webbprogrammering 7.5 hp VT-14 }
\begin{document}
\maketitle
\vspace{-2em}
%\tableofcontents



\section{Introduktion}
I denna laboration kommer vi att arbeta med HTML, CSS, PHP, Javascript och MySQL. Slutresultatet kommer vara en sida där man kan posta gäsboksinlägg. AJAX - Exakt alla requester skall ske med AJAX.

\section{Föreslagen struktur}
Denna laboration består av 4 faser men kommer endast resultera till ett program. Denna laboration kräver ingen inlämning men för att få en inblick om vad som krävs ges ett förslag på hur ni kan strukturera era dokument.
  \begin{itemize}
    \item labb1\_fornamn\_efternamn.zip
      \begin{itemize}
        \item Laboration1 (mapp)
          \begin{itemize}
            \item index.php 
            \item images (mapp med bilder)
            \item js (mapp med javscriptfiler)
	   \item css (mapp med stylesheets)
          \end{itemize}
    \end{itemize}
  \end{itemize}

Observera att hur många filer ni har i varje mapp beror på vad ni använder er av. Läs vidare i instruktionerna för klarare förståelse.




\pagebreak
\section{Uppgifter}
Nedan följer uppgifterna som resulterar i inlämningen ovan.

  \exercise{Refaktorisering}
Innan ni börjar att byta ut alla requesters som tvingar att ladda om sidan är det bra att gå igenom sin kod för att förfina sin lösning.

 \begin{itemize}
            \item Duplicerad kod inom ett dokument.
            \item Duplicerad kod mellan filer.
            \item Variabel- och metodnamn.
  \end{itemize}

  \exercise{AJAX}
Antagligen så har du vid varje funktion anrop till servern användt dig av tekniken att ladda om hela sidan när något förändrar sig. Eftersom ett anrop till servern tar tid så är det en god ide att inte behöva ladda om sidan mer än nödvändigt och särskilt inte om det endast är delar av innehållet på sidan som skall uppdateras. För detta används istället AJAX. 

Er uppgift är att vid varje anrop endast använda er av AJAX och utifall ni dirigeras till en annan sida vid exempelvis att skriva ett inlägg och visa inlägg skall nu detta ske på samma sida, utan att ladda om sidan. 





\end{document}