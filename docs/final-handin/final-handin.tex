\documentclass[12pt]{article}
\usepackage{fullpage}
\usepackage[swedish]{babel}
\usepackage[utf8]{inputenc} % åäö
%\usepackage[T1]{fontenc}
\usepackage{graphicx}
\usepackage{hyperref}
\usepackage{xcolor}
\usepackage{listings}
\usepackage{enumitem}

% No numbering
\setcounter{secnumdepth}{0}

% Counters for tasks & questions
\newcounter{taskcounter}
\setcounter{taskcounter}{0}
\newcounter{stepcounter}
\setcounter{stepcounter}{0}
\newcounter{questioncounter}
\setcounter{questioncounter}{0}

% Exercises
\newcommand{\exercise}[1]{
  \refstepcounter{taskcounter}
  \addcontentsline{toc}{subsection}{Uppgift \thetaskcounter{} #1}
  \vspace{1em}~
  \\\normalfont{\large{\bfseries{\hspace{0.5em}Uppgift \thetaskcounter \hspace{1em}#1}}}\\\\
}

% Remove date
\date{}

\hypersetup{
  colorlinks = true,
  linkcolor = blue,
  citecolor = red
}

\lstset{
  language=[Sharp]C,
  basicstyle=\color[rgb]{0.3,0.3,0.3}\ttfamily,
  keywordstyle=\color[rgb]{0,0.5,0.5},
  numberstyle=\color[rgb]{0.7,0.7,0.7},
  commentstyle=\color[rgb]{0.1,0.5,0.1},
  stringstyle=\color[rgb]{0.6,0.1,0.5},
  backgroundcolor=\color[rgb]{0.95,0.95,0.95},
  showstringspaces=false,
  numbers=left,
  breaklines,
  breakatwhitespace,
}

\title{ Krav för slutgiltig inlämning för laborationerna }

\author{ E-tjänster och webbprogrammering 7.5 hp VT-14 }
\begin{document}
\maketitle
\vspace{-2em}
%\tableofcontents



\section{Introduktion}
Laborationerna i sig är inte obligatoriska. Således finns det inte heller någon inlämning per laboration. Den som utför laborationerna kommer dock att ha ett \textbf{signifikant försprång} när det kommer till denna uppgift. Denna uppgift är obligatorisk och ska redovisas muntligt.

\section{Uppgiften}
Uppgiften går ut på att skapa en sida med kommentarsfunktionalitet. Endast inloggade användare ska kunna posta kommentarer. Således behöver användare kunna registrera sig och logga in. Användare ska även kunna logga ut. Kommentarer listas med avsändare och meddelande. Den användare som är inloggad är den som står som avsändare för en kommentar. Användare måste registreras med E-post lösenord.

Alla fält ska (på ett adekvat sätt) valideras på både klient-, och serversidan. Använderes lösenord får \textbf{inte} sparas som plain-text, utan ska hash:as och saltas. Alla sidor måste vara skyddade ifrån \texttt{SQL injections}.

Kommentarer ska postas genom AJAX (alltså utan att sidan laddas om). Mer funktionalitet \textbf{får} utföras genom AJAX.

Sidan ska se professionell ut. Oseriösa eller slarviga inlämningar godtas ej.






\pagebreak
\section{Föreslagen struktur}
  \begin{itemize}
    \item labb1-fornamn-efternamn (mapp)
    \begin{itemize}
      \item \texttt{index.php}
      \item \texttt{registrar.php} 
      \item \texttt{login.php}  
      \item \texttt{assets} (mapp)
      \begin{itemize}
        \item \texttt{img} (mapp med bilder)
        \item \texttt{js  } (mapp med javscriptfiler)
        \item \texttt{css} (mapp med stylesheets)
      \end{itemize}
    \end{itemize}
  \end{itemize}




\begin{tabular}{ l | c | r }
  php & 2 & 3 \\ \hline
  css & 5 & 6 \\ \hline
  Javascript & 8 & 9 \\
\end{tabular}




\end{document}