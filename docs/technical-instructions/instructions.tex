\documentclass{article}

\usepackage{amssymb}
\usepackage{listings}
\usepackage{comment}
\usepackage[utf8]{inputenc}
\usepackage{color}

\begin{document}

  \title{ Användande av webbservern | E-tjänster och webbprogrammering }
  \author{ Uppsala Universitet }
  \date{ VT-14 }
  \maketitle

  \lstset{language=PHP}

  \section{Åtkomst till webbservern}
    Addressen till webbservern är \texttt{http://ts02.im.uu.se/}. Om du går till den adressen borde du få Apaches default-sida som säger ``It works!". Ditt egna utrymme på webbservern hittar du under \texttt{http://ts02.im.uu.se/student/XXXX}. Där XXXX motsvarar ditt användarnamn. Om en mapp inte finns med ditt användarnamn, skapa den själv.

  \paragraph{}
    Denna mapp hittar du under \texttt{c:/Apache2/htdocs/student/XXXX}. Av säkerhetsskäl så är det dock så att du inte kan hitta C: genom Den här Datorn. Istället behöver du gå via kommandoprompten. Du hittar kommandoprompten under \texttt{Start > All programs > Accessories > Command Prompt}. Via kommandoprompten kan du nu förflytta dig till enhet C: och sedan öppna ovan nämnd mapp.

  \section{Mycket kort om kommandoprompten}
    Här kommer några snabba kommandon du kan använda i kommandoprompten.

    \begin{description}
      \item[\texttt{cd XX}] \hfill \\
      Förflyttar dig till mappen XX genom relativ sökväg. Letar alltså efter mappen i din nuvarande mapp (Tänk: ``change directory'').
      
      \item[\texttt{cd /XX}] \hfill \\
      Förflyttar dig till mappen /XX genom absolut sökväg. Letar alltså efter mappen i nuvarande enhet (ex. C:).

      \item[\texttt{cd ..}] \hfill \\
      Förflyttar dig en mapp uppåt.

      \item[\texttt{C:}] \hfill \\
      Förflyttar dig till enheten med etiketten C.

      \item[\texttt{dir}] \hfill \\
      Listar filer och mappar i nuvarande mapp.

      \item[\texttt{help XX}] \hfill \\
      Genom att skicka ett kommando till help-kommandot får du hjälpinformation för just det kommandon. Fungerar för de flesta kommandon och är mycket användbart.

      \item[\texttt{mkdir XX}] \hfill \\
      Skapa en mapp vid platsen XX.

      \item[\texttt{explorer XX}] \hfill \\
      Öppnar ett vanligt Explorer fönster vid platsen XX. Om du låter XX vara ``.'' (punkt) så tolkar kommandoprompten det som din nuvarande plats.
    \end{description}

  \paragraph{}
    Eftersom Apache körs på \texttt{c:/Apache2/htdocs/} och din mapp ligger under den mappen så kommer alla php filer som lägger i den mappen att kunna köras exekveras av servern.

  \section{Åtkomst till databasen}
  Använd dig av MySQL Administrator. Om du ansluter med ditt användarnamn och lösenord så bör du ha en databas (``schema'') med samma användarnamn som du har fulla rättigheter till.

  \paragraph{}
    \begin{tabular}{l c l}
    Server & : & astrid.dis.uu.se \\
    Port & : & 3306 \\
    Username & : & Se nedan \\
    Password & : & Se nedan \\
    \end{tabular}

  \subsection*{Användaruppgifter}
    \texttt{TODO...} \\


\end{document}
 \\