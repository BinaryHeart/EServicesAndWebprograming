\documentclass[12pt]{article}
\usepackage{fullpage}
\usepackage[swedish]{babel}
\usepackage[utf8]{inputenc} % åäö
%\usepackage[T1]{fontenc}
\usepackage{graphicx}
\usepackage{hyperref}
\usepackage{xcolor}
\usepackage{listings}
\usepackage{enumitem}

% No numbering
\setcounter{secnumdepth}{0}

% Counters for tasks & questions
\newcounter{taskcounter}
\setcounter{taskcounter}{0}
\newcounter{stepcounter}
\setcounter{stepcounter}{0}
\newcounter{questioncounter}
\setcounter{questioncounter}{0}

% Exercises
\newcommand{\exercise}[1]{
  \refstepcounter{taskcounter}
  \addcontentsline{toc}{subsection}{Uppgift \thetaskcounter{} #1}
  \vspace{1em}~
  \\\normalfont{\large{\bfseries{\hspace{0.5em}Uppgift \thetaskcounter \hspace{1em}#1}}}\\\\
}

% Remove date
\date{}

\hypersetup{
  colorlinks = true,
  linkcolor = blue,
  citecolor = red
}

\lstset{
  language=[Sharp]C,
  basicstyle=\color[rgb]{0.3,0.3,0.3}\ttfamily,
  keywordstyle=\color[rgb]{0,0.5,0.5},
  numberstyle=\color[rgb]{0.7,0.7,0.7},
  commentstyle=\color[rgb]{0.1,0.5,0.1},
  stringstyle=\color[rgb]{0.6,0.1,0.5},
  backgroundcolor=\color[rgb]{0.95,0.95,0.95},
  showstringspaces=false,
  numbers=left,
  breaklines,
  breakatwhitespace,
}

\title{ Labb 3 }

\author{ E-tjänster och webbprogrammering 7.5 hp VT-15 }
\begin{document}
\maketitle
\vspace{-2em}
%\tableofcontents



\section{Introduktion}
I denna laboration kommer vi att arbeta i två pass. I det första kommer vi att refaktorera vår applikation (alltså resultatet av tidigare labbar). För att öka nivån av ``kontroll'', struktur och läsbarhet. Vi kommer att refaktorera vår \texttt{HTML}, \texttt{CSS}, \texttt{PHP}, \texttt{JavaScript} såväl som vår \texttt{SQL}. Med andra ord all kod vi skrivit :)


\pagebreak
\section{Föreslagen struktur}
Denna laboration består endast av 1 fas. Detta är en vidarutveckling av den applikation vi byggt i labb 1 och 2. Denna laboration kräver ingen inlämning. Men för att få en uppfattning om hur ett färdigt projekt efter denna labb skulle kunna se ut -- beakta nedan filstruktur.

  \begin{itemize}
    \item lab3-fornamn-efternamn (mapp)
    \begin{itemize}
      \item \texttt{index.php}
      \item \texttt{register.php}
      \item \texttt{register-process.php}
      \item \texttt{login.php}
      \item \texttt{login-process.php}
      \item \texttt{logout-process.php}
      \item \texttt{posts.php}
      \item \texttt{posts-create.php}
      \item \texttt{include} (mapp som innehåller filer som inte bör navigeras till)
      \begin{itemize}
        \item \texttt{bootstrap.php} (Fil som include:ar alla klasser så att det räcker med att ladda in den här filen. Samt utför vitala aktiviteter såsom att t.ex. starta sessionen.
        \item \texttt{views} (mapp som innehåller \texttt{html}-fokuserade filer)
        \begin{itemize}
          \item \texttt{\_header.php}
          \item \texttt{\_footer.php}
          \item \texttt{\_posts-list.php}
          \item \texttt{\_posts-new.php}
          \item \texttt{login.php}
          \item \texttt{posts.php}
          \item \texttt{register.php}
        \end{itemize}
        \item \texttt{models} (mapp som innehåller \texttt{klasser})
        \begin{itemize}
          \item \texttt{db.php}
          \item \texttt{user.php}
          \item \texttt{post.php}
          \item \texttt{authorizer.php}
        \end{itemize}
      \end{itemize}
      \item \texttt{assets} (mapp)
      \begin{itemize}
        \item \texttt{img} (mapp med bilder)
        \item \texttt{js  } (mapp med javscriptfiler)
        \item \texttt{css} (mapp med stylesheets)
      \end{itemize}
    \end{itemize}
  \end{itemize}

Er faktiska filstruktur kan förstås markant variera beroende på vilken strategi ni väljer att lösa problemet med. Samt vilka refaktoreringar ni faktiskt genomför. Läs vidare i instruktionerna för klarare förståelse.



\pagebreak
\section{Uppgifter}
Nedan följer uppgifterna som resulterar i inlämningen ovan.

\exercise{Refaktorering}
Refaktorera din kod så att vi \textbf{åtminstone} uppfyller nedan.

\begin{itemize}
  \item Organisera dina filer i en tydlig och konsistent mappstruktur.
  \item Eliminera duplikation av liknande kod \textbf{inom ett dokument} (i.e.generalisera).
  \item Eliminera duplikation av liknande kod \textbf{emellan dokument} (i.e. generalisera).
  \item Bryt ut långa block av procedurell kod till \textbf{metoder}.
  \item Dela upp metoder som gör flera saker i separata metoder, och ge de tydligare namn.
  \item Låt alla \texttt{variabler}, \texttt{metoder} och \texttt{klasser} ha beskrivande och tydliga namn.
  \item Bryt ut varje databas-query till en egen \texttt{metod}. Använd metodargument för att skicka värden. Exempel: \texttt{createPost(\$author, \$message);}
  \item Om flera databas-query-metoder är liknande. Försök slå ihop dem.
  \item Separera ``printing'' och ``processing''. Låt \textbf{endast} de ``yttersta'' filerna hantera utskrivning av \texttt{html}. Låt inte ``process-filerna'' hantera \texttt{html}. Låt de istället hantera data-processning och business logic. Se förslagen inlämningsstruktur (högst upp) för ett förslag på denna indelning. I förslaget så har man valt att låta alla ``visuella'' filer bo i mappen \texttt{views}. Sedan inkluderar alltså de ``yttersta'' filerna (som inte är \texttt{-process}-filer) motsvarade vy. Notera alltså att den ``yttre'' filen \texttt{posts.php} har en motsvarande fil under \texttt{views/posts.php}. Under mappen \texttt{models} lägger vi istället klasser eller funktioner som hanterar affärslogik. Dessa filer skriver aldrig ut någonting till skärmen. Detta betyder att de ``yttre'' filerna kan fokusera på att vara ``spindeln i nätet'' och komponera ihop alla delar rätt. Så att om man t.ex. öppnar \texttt{posts.php} så ser den till att först inkludera den fil ifrån mappen \texttt{models} som förbereder så att vi har all data (poster) vi vill ha i en variabel. Sedan inkluderar vi rätt vy ifrån mappen \texttt{views}. Vilken använder sig av den variabel med data vi tidigare satte. Vyn printar helt enkelt ut html.
  \item Blanda \texttt{html} och \texttt{php} på ett tydligt och konsistent sätt. Du måste inte följa strukturen som är föreslagen i detta dokument. Men det är viktigt att du \textbf{har} en struktur.
   \item Nu när vi fått ordning på struktur och förhoppningsvis fått det lättarbetat och öppen för förändring skall ni nu byta ut javascripten till att använda er av jQuery istället. 
\end{itemize}




\end{document}
