\documentclass[12pt]{article}
\usepackage{fullpage}
\usepackage[swedish]{babel}
\usepackage[utf8]{inputenc} % åäö
%\usepackage[T1]{fontenc}
\usepackage{graphicx}
\usepackage{hyperref}
\usepackage{xcolor}
\usepackage{listings}
\usepackage{enumitem}

% No numbering
\setcounter{secnumdepth}{0}

% Counters for tasks & questions
\newcounter{taskcounter}
\setcounter{taskcounter}{0}
\newcounter{stepcounter}
\setcounter{stepcounter}{0}
\newcounter{questioncounter}
\setcounter{questioncounter}{0}

% Exercises
\newcommand{\exercise}[1]{
  \refstepcounter{taskcounter}
  \addcontentsline{toc}{subsection}{Uppgift \thetaskcounter{} #1}
  \vspace{1em}~
  \\\normalfont{\large{\bfseries{\hspace{0.5em}Uppgift \thetaskcounter \hspace{1em}#1}}}\\\\
}

% Remove date
\date{}

\hypersetup{
  colorlinks = true,
  linkcolor = blue,
  citecolor = red
}

\lstset{
  language=[Sharp]C,
  basicstyle=\color[rgb]{0.3,0.3,0.3}\ttfamily,
  keywordstyle=\color[rgb]{0,0.5,0.5},
  numberstyle=\color[rgb]{0.7,0.7,0.7},
  commentstyle=\color[rgb]{0.1,0.5,0.1},
  stringstyle=\color[rgb]{0.6,0.1,0.5},
  backgroundcolor=\color[rgb]{0.95,0.95,0.95},
  showstringspaces=false,
  numbers=left,
  breaklines,
  breakatwhitespace,
}

\title{ Labb 2 }

\author{ E-tjänster och webbprogrammering 7.5 hp VT-14 }
\begin{document}
\maketitle
\vspace{-2em}
%\tableofcontents



\section{Introduktion}
I denna laboration kommer vi att arbeta med HTML, CSS, PHP, Javascript och MySQL. Denna laboration kommer att resultera till att skapa en sida som man är tvungen att logga in till. Ni kommer att bekanta er med både HTTPs get och post, Hasning och saltning. egen saltning och PHPs hash (PHP bcrypt) till Gästboken. Ni skall alltså skapa en inloggning till er gästbok som gjordes i förra laborationen. 



\section{Föreslagen struktur}
Denna laboration består av 4 faser men kommer endast resultera till ett program. Denna laboration kräver ingen inlämning men för att få en inblick om vad som krävs ges ett förslag på hur ni kan strukturera era dokument.
  \begin{itemize}
    \item labb1\_fornamn\_efternamn.zip
      \begin{itemize}
        \item Laboration1 (mapp)
          \begin{itemize}
            \item index.php 
            \item login.php
	   \item registration.php
            \item images (mapp med bilder)
            \item js (mapp med javscriptfiler)
	   \item css (mapp med stylesheets)
          \end{itemize}
    \end{itemize}
  \end{itemize}

Observera att hur många filer ni har i varje mapp beror på vad ni använder er av. Läs vidare i instruktionerna för klarare förståelse.


\pagebreak
\section{Uppgifter}
Nedan följer uppgifterna som resulterar i inlämningen ovan.


  \exercise{Designa databasen}
Ni behöver som har hand om användarna som finns registrerade till denna fantastiska gästbok. Denna tabell behöver ha hand om minst användarnamn, lösenord men även att kunna ha en kolumn för salt. 

 \exercise{Skapa en registreringssida}

Skapa en sida som har ett formulär som kan skapa användare. 

  \begin{enumerate}
    \item Börja med att skapa registraion.php 
    \item Skapa ett formulär där användaren kan skriva in sina uppgifter
    \item Skapa en function som randomiserar en sträng som skall användas som salt till lösenordet.
    \item Kryptera salt+ det lösenord som användaren valt.
    \item Lägg in allt i databasen och ge användaren information om registreringen lyckades eller inte. 
   \end{enumerate}

 \exercise{Skapa inloggningssida} 
 Skapa en inloggningssida för er applikation. 

 \begin{enumerate}
    \item Börja med att skapa login.php
    \item Skapa ett formulär där användaren kan skriva in sina inloggningsuppgifter
    \item Hämta saltet från databasen
    \item Kryptera salt+ lösenord och matcha mot det lösenord som finns i databasen. 
    \item Sätt en session.
   \end{enumerate}

 \exercise{Uppdatera indexsidan} 
I denna section skall ni nu se till att följande kan ske. Ni skall endast tillåta att användare som är inloggade skall kunna se sidan. Om man skulle skriva in sidan utan att vara inloggad skall man antingen få information om att de måste logga in eller transporteras till inloggningssidan på en gång. 

Sedan skall ni skapa en knapp eller länk där användaren kan logga ut.  

 \exercise{Smycka sidan} 
Se nu till att sidan håller ett enhetligt intryck i sin design. Med det menas att det skall synas att sidan håller samma stilmall.

 \exercise{Förslag på uttökningar} 
\begin{itemize}
            \item Se till att man inte får skapa användare med samma användarnamn.
	   \item  Kontrollera att användaren anger starka lösenord. 
          \end{itemize}


\end{document}